\chapter{Introducción}\label{cap:capitulo_1}
%---------------------------------------------------------------------------
La sociedad actual, está completamente digitalizada, pero carece de conciencia digital que permite disminuir los riesgos: La falta de conocimiento relacionado con las tendencias actuales del Internet de las Cosas (IoT), las plataformas cloud, los servicios en la nube, las redes sociales y el uso de aplicaciones Fintech. (De este modo se atraen a los ciberdelincuentes para aprovecharse y ampliar su modus operandi). Aun así, toman ventaja de la información personal publicada en redes sociales, además de que utilizan ataques para intervenir y captar datos que pueden utilizar en contra de la víctima para sacar beneficio. Aun cuando son desplegadas y adoptadas estás tecnologías se requiere de un personal consiente como: primeria línea de defensa ante ataques; ya que los sistemas por si solos a pesar de contar con mecanismos de tecnologías preventivas y para la detección (son los usuarios los cuales manipulan, configuran y utilizan las tecnologías). \\
El creciente uso de los medios digitales en diversas áreas cotidianas de la vida ha hecho que los ciudadanos estén más conectados a lo que sucede en su entorno, estos utilizan tecnologías para comunicarse por múltiples medios sociales, controlar remotamente equipos, maquinarias y realizar transacciones. Este nuevo ecosistema digital donde concurren las interacciones sociales contiene una creciente ola de riesgos que pueden moldear el comportamiento en masa, influir, manipular y provocar daños con ataques en el ecosistema digital en la llamada nueva ciudadanía digital \cite{A2020}. En necesaria para la sociedad en general que los desarrollos e implementaciones de tecnologías se orienten a innovaciones inclusivas que mejoren el bienestar social, algo que es un reto para países menos desarrollados en el ámbito cultural tecnológico creando brechas de conocimiento que provocan afectaciones en la seguridad personal, social, empresarial y nacional. Como lo menciona  \cite{Nadeesha2021}, casos como el ciberacoso, el fraude, suplantación de identidad, filtración de datos y perdidas, infiltración en sistemas críticos pueden generar caos social, por lo que es necesario educar desde temprano sobre seguridad cibernética, otros autores que han realizado investigaciones sobre conciencia digital han identificado los grupos de estudio tienen un bajo nivel de conocimiento sobre malwares, técnicas de phishing, ataques de fuerza bruta y uso de contraseñas inseguras \cite{Eslavova2019}. \\
Esta investigación busca conocer el nivel de conciencia sobre riesgos cibernéticos en un grupo focal y plantea una propuesta a través de un prototipo de asistencia por lenguaje natural a través de texto por medio de modelo GPT adoptado a pequeña escala y ajustado a temas de que apoyen a la concienciación sobre seguridad informática y navegación segura en internet.
 
\section{Formulación de hipótesis y objetivos} \label{section:thesis organization} % you can assign a unique label to each chapter or section for crossreferencing

    \begin{itemize}
        \item En Panamá existe conciencia sobre los riesgos y vulnerabilidades en el ciberespacio.
        \item Mediante un modelo GPT entrenado con un Dataset se puede dar un mejor enfoque para concientizar y comunicar las amenazas y brindar recomendaciones de mitigación.
        \item ¿Un asistente puede concienciar sobre la seguridad digital y sobre la importancia de la seguridad y del uso de técnicas que permitan aplicar buenas ‘prácticas de navegación en la sociedad informática en los medios electrónicos digitales?

    \end{itemize}

%/\subsection{This is a subsection}
 {\normalsize \bf Palabras Clave: }
    {\normalsize Asistente digital, Seguridad Informática, Modelo GPT, Cultura de seguridad digital.}

%---------------------------------------------------------------------------

%---------------------------------------------------------------------------