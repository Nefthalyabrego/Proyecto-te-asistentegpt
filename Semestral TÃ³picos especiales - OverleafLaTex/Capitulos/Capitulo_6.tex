%\chapter{Conclusiones}\label{cap:capitulo6}
%---------------------------------------------------------------------------
%\section{Reflección}\label{seccion:reflection}


\section{Conclusión}\label{seccion:conclusion}

%---------------------------------------------------------------------------
Evaluar cómo es la cultura en temas digitales en Panamá, es una tarea difícil ya que sería comprender su nivel al panorama actual e identificación en áreas de mejoras. Esto incluye una exhaustiva evaluación a la competencia en ámbitos tecnológicos, el acceso a distintas tecnologías, el uso con responsabilidad del Internet y la seguridad en línea de la población. 
Identificar las necesidades y los desafíos que con lleva, durante el tiempo de evaluación, es de mucha importancia. Ya que determinara esas necesidades y desafíos muy específicos en correlación a nuestra cultura digital y la navegación de manera segura en Internet. Esto permitirá que se establezca objetivos de manera clara y de ese modo desarrollar estrategias mucho más adecuada a la hora de abordar.

El Diseñar un modelo para asistir al tema de “seguridad informática”, puede llegar a ser una herramienta muy eficaz para tratar de lograr el objetivo y es concientizar en el ámbito de navegación segura en Internet. NanoGPT podrá brindar ayuda en recomendaciones, recursos educativos, ayudar a comprender riesgos y adoptar buenas prácticas.

Personalizar el asistente, es una tarea complicada y es recomendable crear un diseño. De modo que se pueda adaptar a las necesidades y características de cada uno de los usuarios en Panamá. De manera que pueda incluir la personalización: en cuanto a contenido, el usar ejemplos, escenas que sean relevantes a el contexto local y tomar a consideraración aspecto culturales y lingüísticos.
