%\chapter{Desarrollo de propuesta de prototipo}\label{cap:capitulo4}

%---------------------------------------------------------------------------
\subsection{Creación de conjunto de datos de entrenamiento y adaptación de nanoGPT}\label{section:Creación de conjunto de datos con fuentes de internet} 
El conjunto de datos contiene recomendaciones y prácticas de seguridad clasificadas por categorías como amenazas, ataques y malware.El conjunto es procesado desde un archivo CSV con las columnas y registros. Se realiza la extración de datos con la librería pandas a un dataframe o estructura de datos para el posterior procesamiento.\cite{Reiss2021}
\begin{figure}[H]
   \centering % figure is centered on the page
       \includegraphics[width=0.7\linewidth]{./doc/Conjunto de datos etiquetado.png} 
   \caption{Características del conjunto de datos etiquetados \cite{}}
  \label{figure:Conjunto de datos}  % assign a unique label to each figure 
\end{figure}
\begin{figure}[H]
   \centering % figure is centered on the page
       \includegraphics[width=0.8\linewidth]{./doc/02-cr.png} 
   \caption{Configuración para extraer el conjunto de datos en un dataframe y preparación de los datos.  \cite{}}
  \label{figure:Extraxión de datos del csv}  % assign a unique label to each figure 
\end{figure}
La etiquetación de los registros permite que el modelo pueda identificar la categoría de clasificación, figura \ref{figure:Conjunto de datos}. Este conjunto se carga para la tokenización, proceso que permite dividir las cadenas en piezas para crear tokens.[25] El entrenamiento con NanoGPT creando las carpetas que albergan el dataset en formato CSV, luego se procesa para el entrenamiento con iteraciones repetidas.[27]
\begin{figure}[H]
   \centering % figure is centered on the page
       \includegraphics[width=0.8\linewidth]{./doc/03-cr.png} 
   \caption{procesamiento crea  ficheros de validación, entrenamiento y un meta modelo de apoyo.  \cite{}}
  \label{figure:Etapa de encoder}  % assign a unique label to each figure 
\end{figure}
\begin{figure}[H]
   \centering % figure is centered on the page
       \includegraphics[width=0.8\linewidth]{./doc/04-cr.png} 
   \caption{Configuraciones y parámetros de entrenamiento que crea los directorios de salida para el modelo entrenado.  \cite{}}
  \label{figure:Configuraciónes de parámetros}  % assign a unique label to each figure 
\end{figure}
%---------------------------------------------------------------------------
\subsection{Configuración de los parámetros del código y entrenamiento}\label{section:Configuración y entrenamiento} 
Posteriormente se configuran los parámetros del código para entrenar con los datos un modelo sobre nanoGPT con ciclos de entrenamiento. Para ello se crea finalmente y se evalúan las respuestas y se propone una interfaz para el prototipo con un enfoque MVC.
%-------------------------------------------------------------------------------
\subsubsection{Configuración 1}\label{section:Configuración 1} 
\begin{itemize}
        \item   Objetivo: Ejecutar el entrenamiento de manera más rápida al reducir el número máximo de iteraciones y ajustar otros parámetros.
        \item   Explicación: Esta configuración aumenta el número de capas (n-layer) y cabezas de atención (n-head) a 12 para proporcionar una capacidad de representación más alta. El tamaño del lote (batch-size) se incrementa a 64 para aprovechar mejor los recursos de la CPU. Se reduce el número máximo de iteraciones (max-iters) a 1000 para acelerar el entrenamiento. El número de iteraciones para reducir la tasa de aprendizaje (lr-decay-iters) se establece en 1000. Se mantiene un tamaño de bloque (block-size) de 128 y una dimensión del espacio de embedding (n-embd) de 512 para mantener una representación adecuada del texto. La probabilidad de dropout (dropout) se establece en 0.2 para regularizar el modelo durante el entrenamiento.
    \end{itemize}
    \begin{figure}[H]
   \centering % figure is centered on the page
       \includegraphics[width=0.65\linewidth]{./rp/15-cp.png} 
   \caption{Proceso de entrenamiento de modelo 1\cite{}}
  \label{figure:Entrenamiento de modelo 1}  % assign a unique label to each figure 
\end{figure}
%------------------------------------------------------------------------------
\subsubsection{Configuración 2}\label{section:Configuración de los parámetros del código} 
    \begin{itemize}
        \item   Objetivo: Evaluar el rendimiento y la eficiencia de un modelo ligero en el conjunto de datos proporcionado.
        \item   Explicación: Esta configuración se centra en entrenar un modelo con una arquitectura más simple y menos parámetros para tareas de clasificación en el conjunto de datos de seguridad que permitan acercar. Se utiliza un tamaño de bloque más pequeño (block-size) para limitar el contexto de los caracteres anteriores. 
        \item   Se reduce el número de capas (n-layer) y cabezas de atención (n-head) para disminuir la complejidad del modelo. El tamaño del lote (batch-size) también se reduce para adaptarse a recursos limitados.
    \end{itemize}
\begin{figure}[H]
   \centering % figure is centered on the page
       \includegraphics[width=0.65\linewidth]{./rp/05-cp.png} 
   \caption{Configuraciones de entrenamiento de modelo 2  \cite{}}
  \label{figure:Configuraciones modelo entrenamiento 2}  % assign a unique label to each figure 
\end{figure}
\begin{figure}[H]
   \centering % figure is centered on the page
       \includegraphics[width=0.65\linewidth]{./rp/06-cp.png} 
   \caption{Proceso de entrenamiento de modelo 2\cite{}}
  \label{figure:Proceso de entrenamiento del modelo 2}  % assign a unique label to each figure 
\end{figure}
%----------------------------------------------------------------------------
\subsection{Validación del modelo Configuración 1}\label{section:Validación de prompt}
\subsubsection{ Prueba 1}\label{section:Validación del prototipo}
    \begin{itemize}
        \item   Top-k = 100
        \item   Temperatura = 1.2
        \item   Ma-new-tokens = 500
        \item    Prompt = python3 sample.py --out-dir=out-assistmade-char --device=cpu --start=" phising protection tips"
    
\begin{figure}[H]
   \centering % figure is centered on the page
       \includegraphics[width=0.65\linewidth]{./rp/16-cp.png} 
   \caption{Resultados de la Prueba 1\cite{}}
  \label{figure:Prueba1}  % assign a unique label to each figure 
\end{figure}
        \item   Prompt = python3 sample.py --out-dir=out-assistmade-char --device=cpu --start="safe internet browsing "
\begin{figure}[H]
   \centering % figure is centered on the page
       \includegraphics[width=0.65\linewidth]{./rp/17-cp.png} 
   \caption{Resultados de la Prueba 2\cite{}}
  \label{figure:Prueba2}  % assign a unique label to each figure 
\end{figure}
        \item   Prompt = python3 sample.py –out-dir=out-assistmade-char --device=cpu --start="computer security tips"
\begin{figure}[H]
   \centering % figure is centered on the page
       \includegraphics[width=0.65\linewidth]{./rp/18-cp.png} 
   \caption{Resultados de la Prueba 3\cite{}}
  \label{figure:Resultado prueba 3}  % assign a unique label to each figure 
\end{figure}
        \item   Prompt = python3 sample.py --out-dir=out-assistmade-char --device=cpu --start="prevent identity theft"
\begin{figure}[H]
   \centering % figure is centered on the page
       \includegraphics[width=0.65\linewidth]{./rp/19-cp.png} 
   \caption{Resultados de la Prueba 4\cite{}}
  \label{figure:Resultado prueba 4}  % assign a unique label to each figure 
\end{figure}
\end{itemize}
%-----------------------------------------------------------------------------   
\subsection{Validación del modelo Configuración 2}\label{section:Validación Modelo 2}
\subsubsection{ Prueba 1}\label{section:Prueba 1 config 2}
    \begin{itemize}
        \item   Top-k = 1
        \item   Temperatura = 10
        \item   Ma-new-tokens = 500
        \item    Prompt = python3 sample.py --out-dir=out-assistmade-char --device=cpu --start=" phising protection tips"
    \end{itemize}
\begin{figure}[H]
   \centering % figure is centered on the page
       \includegraphics[width=0.65\linewidth]{./rp/07-cp.png} 
   \caption{Resultados de la Prueba 1\cite{}}
  \label{figure:Result prueba 1 mol 2}  % assign a unique label to each figure 
\end{figure}
\subsubsection{ Prueba 2}\label{section:Prueba2}
    \begin{itemize}
        \item   Top-k = 0.8
        \item   Temperatura = 10
        \item   Ma-new-tokens = 500
        \item   Prompt = python3 sample.py --out-dir=out-assistmade-char --device=cpu --start="safe internet browsing"
    \end{itemize}
\begin{figure}[H]
   \centering % figure is centered on the page
       \includegraphics[width=0.65\linewidth]{./rp/08-cp.png} 
   \caption{Resultados de la Prueba 2\cite{}}
  \label{figure:Result prueb 2 mol 2}  % assign a unique label to each figure 
\end{figure}
\subsubsection{ Prueba 3}\label{section:Prueba 3 mol 2}
    \begin{itemize}
        \item   Top-k = 100
        \item   Temperatura = 1.2
        \item   Ma-new-tokens = 500
            \item   Prompt = python3 sample.py --out-dir=out-assistmade-char --device=cpu --start=" phising protection tips"
            \begin{figure}[H]
              \centering % figure is centered on the page
                  \includegraphics[width=0.65\linewidth]{./rp/09-cp.png} 
              \caption{Resultados de la Prueba 3.1\cite{}}
            \label{figure:Result Prueba 3 mod 2}  % assign a unique label to each figure 
            \end{figure}
            \item   Prompt = python3 sample.py --out-dir=out-assistmade-char --device=cpu --start="safe internet browsing"
            \item   Prompt = python3 sample.py --out-dir=out-assistmade-char --device=cpu --start="computer security tips"
            \begin{figure}[H]
              \centering % figure is centered on the page
                  \includegraphics[width=0.65\linewidth]{./rp/10-cp.png} 
              \caption{Resultados de la Prueba 3.2 y 3.3\cite{}}
            \label{figure:Resultado 3.2}  % assign a unique label to each figure 
            \end{figure}
            \item   Prompt = python3 sample.py --out-dir=out-assistmade-char --device=cpu --start="computer security tips"
            \begin{figure}[H]
              \centering % figure is centered on the page
                  \includegraphics[width=0.65\linewidth]{./rp/11-cp.png} 
              \caption{Resultados de la Prueba 3.4\cite{}}
            \label{figure:Resultado 3.4}  % assign a unique label to each figure 
            \end{figure}
    \end{itemize}
\subsubsection{ Prueba 4}\label{section:Adaptación de modelo nanoGPT}
    \begin{itemize}
        \item   Top-k = 100
        \item   Temperatura = 1.1
        \item   Ma-new-tokens = 500
            \item   Prompt = python3 sample.py --out-dir=out-assistmade-char --device=cpu --start=" phising protection tips"
            \begin{figure}[H]
              \centering % figure is centered on the page
                  \includegraphics[width=0.65\linewidth]{./rp/12-cp.png} 
              \caption{Resultados de la Prueba 4.1\cite{}}
            \label{figure:Resultado 4 1}  % assign a unique label to each figure 
            \end{figure}
            \item   Prompt = python3 sample.py --out-dir=out-assistmade-char --device=cpu --start="computer security tips"
            \begin{figure}[H]
              \centering % figure is centered on the page
                  \includegraphics[width=0.65\linewidth]{./rp/13-cp.png} 
              \caption{Resultados de la Prueba 4.2\cite{}}
            \label{figure:Resultado prueba 4 2}  % assign a unique label to each figure 
            \end{figure}
            \item   Prompt = python3 sample.py --out-dir=out-assistmade-char --device=cpu --start="computer security tips"
            \begin{figure}[H]
              \centering % figure is centered on the page
                  \includegraphics[width=0.65\linewidth]{./rp/14-cp.png} 
              \caption{Resultados de la Prueba 4.3\cite{}}
            \label{figure:Result prueba 4}  % assign a unique label to each figure 
            \end{figure}
    \end{itemize}
%---------------------------------------------------------------------------
%------------------------------------------------------------------------------

