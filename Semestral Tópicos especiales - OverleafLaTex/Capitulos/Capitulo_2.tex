\chapter{Marco Teórico}\label{cap:capitulo_2}
%---------------------------------------------------------------------------

\section{Cultura, sociedad y conciencia digital.}\label{section:Cultura, sociedad y conciencia digital.}
La ciudadanía digital es la adopción del uso de sistemas de información y tecnológico para la sociedad, a través de los cuales realizan transacciones diarias con su entorno. Esto permite que todos los ciudadanos puedan consumir y utilizar los servicios públicos a través de medios digitales\cite{Popkova2020}.
A medida que la sociedad y las ciudades pasan a un enfoque inteligente donde todo está digitalizado aumentan más los desafíos de seguridad protección y privacidad. Muchas ciudades utilizan medidores inteligentes, cámaras de vigilancia por internet, pagos de servicios públicos, redes wifi-nacionales que logran una conectividad y alcance mayor al acceso a internet. Toda la información generada en la sociedad está siendo digitalizada \cite{Konkolewsky2017}. \\
La ciudadanía digital se puede llevar a través de comunidades en internet en donde el individuo adquiere una identidad, derechos, obligaciones que hacen que su participación a través de esta sea válida para la interacción social con su entorno de forma libre\cite{Popkova2020}. \\
La conciencia digital es el punto de partida para una cultura digital sólida, diversos estudios demuestran que la capacitación sobre la seguridad informática es una excelente manera de comunicar y concientizar a la población sobre las amenazas digitales. En el estudio realizado por \cite{}, se demuestra esto a través de la investigación de un grupo de niños y jóvenes a los cuales se les proporcionó información sobre las amenazas en internet, casos de estudio e historias. Otros estudios demuestran como las empresas adoptan este sistema de capacitaciones para asegurar la infraestructura y sus activos para garantizar la continuidad y cultura empresarial digital logrando mitigar las amenazas cibernéticas con este enfoque. \cite{Li2021}\\
Al igual que una empresa, los países deben buscar proteger a los ciudadanos, los programas educativos, las implementaciones curriculares de tecnología permiten hacer que las generaciones venideras se conviertan en buenos ciudadanos digitales y usuarios de internet. Es complicado lograr esto en la práctica ya que, cada generación, grupo social, puede ser visto desde diferentes ángulos con respecto a el ámbito tecnológico, este aspecto generacional y profundización de internet en las generaciones venideras pueden hacerlas más propensas a engaños, perdida de sus datos y renuncia a su información con tal de aprovechar los beneficios de los servicios que utilizan\cite{Francis2022}. \\
%---------------------------------------------------------------------------
\section{Cultura digital y ciberseguridad en Panamá}\label{section:Cultura digital y ciberseguridad en Panamá}
En la investigación de \cite{Graell2022}, se indaga sobre el ciudadano panameño en aspectos de brechas digitales e informática educativa. En la investigación se realizó un proceso exploratorio para ver la adopción y cotidianidad de uso de tecnologías sobre materia digital. En Panamá se tiene un alcance de la red nacional de internet que permite cerrar brechas, pero aún existen retos para romper las desigualdades de acceso y participación a través de la red, algo se deslumbró con la pandemia como lo menciona \cite{} en la educación, en el acceso a la información por medios digitales obligados por la pandemia. Muchas de las instituciones han transformado sus servicios a enfoques digitales que permiten el rápido acceso a sus ciudadanos. Plataformas como Panamá Digital y muchas otras implementaciones a nivel educativo han aumentado la forma en que la sociedad interactúa con las instituciones educativas, financieras etc. Los retos de las brechas sociales y las capacidades relacionadas para crear conciencia digital entre diversas culturas internas, grupos sociales dificultan la capacitación a nivel de País para mitigar problemas y afectaciones a ciudadanos con carentes conocimientos que acceden a estas plataformas de instituciones públicas. Según \cite{Graell2022}, el ciudadano digital debe poseer educación en tecnología, estándares de conducta en medios electrónicos, participación electrónica, responsabilidad, libertades e inclusión, conciencia digital sobre los riesgos. Este conjunto de capacidades del ciudadano digital hace que este posea capacidad para actuar con conciencia y libertad siendo responsabilidad sobre los medios digitales, este espacio de sociedad virtual y cultura permite que los ciudadanos mantengan estándares de conducta y conciencia que permite la paz social resultado del buen el uso de la tecnología. \\
Por otro lado, según la AIG, en Panamá no se escapa de los ciberataques, en el 2021 se registraron en el País más de 3.2 millones de intentos, durante la pandemia las cifras aumentaron por el trabajo remoto, las clases en línea, pagos a empresas a través de medios virtuales etc. Estas cifras colocan a Panamá en el top 10. Si las empresas, personas y el gobierno no implementan medidas y una cultura de ciberseguridad estos ataques pueden generar alteración en la vida cotidiana y en la sociedad \cite{Train-employees}.

%---------------------------------------------------------------------------
\section{ Delitos informáticos en Panamá y leyes}\label{section:Delitos informáticos en Panamá y leyes}
En Panamá existen leyes que determinan claramente los delitos cibernéticos como se menciona en el TITULO VIII del Código Penal sobre los delitos contra la seguridad jurídica de los medios electrónicos. En él se mencionan los delitos contra la seguridad informática y otros cuatro artículos,289,290,291 y 292. \\
Estas leyes apoyan las garantías constitucionales o legales en Panamá y penalizan o sancionan a aquellos que ejecuten actos delictivos por medios electrónicos o tecnológicos. Sin embargo, la sociedad no tiene cultura de denuncia cuando se es víctima de estos actos delictivos y las leyes actuales no persiguen estos delitos verdaderamente \cite{Francis2022}.

%---------------------------------------------------------------------------
\section{Revisión de la literatura}\label{section: Revisión de la literatura}
\subsection{Asistentes digitales}
Los asistentes digitales se han implementado en muchas industrias, empresas, departamentos para establecer conversaciones con las personas. Estos asistentes son capaces de comprender la entrada del usuario y responder bajo el contexto. La inteligencia artificial ha empoderado a estos asistentes haciéndolos más robustos y fuertes para mantener la interacción \cite{Urribarri2022}.
\subsection{Procesamiento del lenguaje natural e inteligencia artificial}\label{section: Revisión de la literatura}
En los últimos años hemos visto como la disciplina aplicada de la inteligencia artificial han incrementado los casos de uso y beneficiado a múltiples sectores haciendo que se mejore la capacidad de análisis y procesamiento con inteligencia que simula a la humana. Los asistentes virtuales con IA es uno de estos casos de uso en donde se aplica tecnología de procesamiento de lenguaje natural; esta técnica permite que un equipo computacional pueda analizar e interpretar el idioma humano a través de algoritmos que permiten realizar procesos para comprender las estructuras textuales. Esta área es un subconjunto de técnicas utilizadas para generar algoritmos de inteligencia artificial que se acerquen a las capacidades humanas. La inteligencia artificial en general propone muchas oportunidades de aplicación en contextos de la vida diaria.
\subsection{Modelos y herramientas para generar conversaciones fluidas}\label{section: Revisión de la literatura}
Los modelos transformadores generativos pre-entrenados(GPT) utilizan técnicas de aprendizaje profundo, procesamiento del lenguaje natural con redes neuronales   para la creación de texto, imágenes, voz, análisis y clasificación. Cuando son utilizados para la generación de texto, estos se basan en conocimientos que adquieren durante un entrenamiento con grandes cantidades de datos lingüísticos. 
Durante el proceso de codificación del texto de entrada (encoder) se crea una representación numérica vectorial que puede ser procesada por la máquina para posteriores decodificaciones, durante este proceso el texto es dividido en partes llamadas tokens los cuales son vectorizados y dados de entrada a la red neuronal.
Cuando se inicia el proceso de entrenamiento la red neuronal ajusta sus pesos como perillas que ajustan probabilísticamente el modelo para dar el texto más acorde a la secuencia de palabras.\cite{Alex2023}, \cite{AWS2022}

El creciente uso de los modelos de lenguaje natural ha mejorado y dado muchas aplicaciones prácticas de procesamiento del lenguaje natural. Estos modelos requieren un corpus de texto de entrenamiento seguido de ajustes que permitan ajustar el modelo para responder y realizar tareas específicas, el modelo GPT-3 tiene 175 mil millones de parámetros y fue entrenado con un amplio contenido de texto web \cite{Brown2020}. 
Existen marcos de trabajo configurables que permiten crear soluciones GPT rápidamente, una de estas es nanoGPT que admite crear modelos con grandes cantidades de parámetros como GPT-2, desde una estructura de código limpia y de pocas líneas. 
%---------------------------------------------------------------------------
\section{Estudios relacionados}\label{section: Estudios relacionados}
La inteligencia artificial y sus sub-ramas proponen aplicaciones que pueden ser orientarse a soluciones para generar más inclusión social y desarrollo cultura de un país. \\
Actualmente no hay estudios que basen modelos de asistentes orientados en la asistencia a la educación sobre ciberseguridad y conciencia digital. Los estudios relacionados se basan en modelos orientados a asistentes comerciales como el realizado por \cite{Eleannor2022}, que implemento un sistema web basado en plataforma DialogFLow que trabaja con el modelo BERT para interpretar el texto, consistió en un asistente web para apoyo a la atención al cliente a través de una interfaz web. El estudio muestra una arquitectura de cliente servidor que permite a través de una interfaz interactuar con el cliente final para responder preguntas relacionadas al comercio. El estudio muestra como el modelo de DialogFlow permite utilizar los algoritmos de NPL para responder a preguntas no establecidas previamente, al estar utilizando modelos NPL se puede interpretar la pregunta del usuario y que el modelo responda con base a el conjunto de datos con el que fue entrenado.
%---------------------------------------------------------------------------
\section{Objetivos del proyecto}\label{section:Objetivos del proyecto}
\subsection{Objetivo general}\label{section:Objetivo general}
Evaluar el nivel de conciencia digital en Panamá mediante la propuesta  de un prototipo de asistente digital basado en un modelo GPT, entrenado con datos sobre seguridad informática.
\subsection{Objetivos específicos}\label{section:Objetivos especificos}
 \begin{enumerate}
        \item Construir del conjunto de datos de entrenamiento relacionado a vulnerabilidades comunes, consejos de mitigación y de buenas prácticas de navegación y uso de medios digitales.
        \item Ajustar un modelo GPT mediante la utilización del conjunto de datos para asistir con respuestas relevantes sobre seguridad informática, consejos de mitigación de vulnerabilidades, buenas prácticas de navegación.
        \item Entrenar el modelo GPT, utilizando un conjunto de datos optimizados, relacionados a los riesgos de la navegación en Internet. 
    \end{enumerate}
    
%---------------------------------------------------------------------------
